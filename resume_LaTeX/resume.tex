\documentclass[a4paper,11pt]{article}

% Пакеты
\usepackage[utf8]{inputenc}
\usepackage[T1]{fontenc}
\usepackage{lmodern}
\usepackage[russian]{babel}
\usepackage{hyperref}
\usepackage{enumitem}
\usepackage{geometry}
\usepackage{fontawesome5}
\usepackage{titlesec}
\usepackage{xcolor}

% Настройка страницы
\geometry{
  a4paper,
  left=2cm,
  right=2cm,
  top=2cm,
  bottom=2cm
}

% Убираем номера страниц
\pagestyle{empty}

% Настройка заголовков
\titleformat{\section}
  {\Large\bfseries}
  {}{0em}
  {}
  [\titlerule]

\titlespacing{\section}
  {0pt}
  {12pt}
  {8pt}

% Настройка гиперссылок
\hypersetup{
  colorlinks=true,
  urlcolor=blue
}

% Начало документа
\begin{document}

% Заголовок резюме
\begin{center}
  {\Huge\textbf{Даниил Дорохов}}\\[0.3cm]
  \faEnvelope\ Telegram: \href{https://t.me/yourusername}{@yourusername} \quad
  \faGithub\ GitHub: \href{https://github.com/yourusername}{github.com/yourusername}
\end{center}

% Профессиональный профиль
\section{Профессиональный профиль}
Full stack developer с практическим опытом разработки веб-приложений и Telegram-решений,
а также обширным background в сфере manual и automated testing [3 года]. Специализируюсь на
создании и интеграции кросс-платформенных решений с использованием современных техноло-
гий. Мой опыт в тестировании на различных платформах (Desktop, Mobile,
Web) обеспечивает уникальный подход к разработке качественных и отказоустойчивых реше-
ний. Активно развиваюсь в направлении full stack разработки, с особым интересом к созданию
Telegram-приложений и ботов.

% Образование
\section{Образование}
\begin{itemize}[leftmargin=*]
  \item \textbf{Российский экономический университет имени Г.В. Плеханова}, Москва \hfill 2023 -- н.в.\\
  Бакалавриат, направление "Бизнес-информатика"\\
  2 курс

\end{itemize}

% Опыт работы
\section{Опыт работы}
\begin{itemize}[leftmargin=*]
  \item \textbf{QA-инженер}, ООО "Avroid" \hfill 12.03.2024 -- н.в.
  \begin{itemize}
    \item Разработка и интеграция Telegram-решений для автоматизации рабочих процессов
    \item Проведение тестирования API с использованием Postman и разработка документации для API
    \item Создание и поддержка коллекции автотестов в Postman
    \item Автоматизация тестирования на уровне API с использованием Python и библиотеки requests
    \item Развертывание тестовых сред с использованием Docker и Aurora SDK
  \end{itemize}

  \item \textbf{QA-инженер}, ООО "Новые Облачные Технологии" ("МойОфис") \hfill 10.01.2023 -- 11.03.2024
  \begin{itemize}
    \item Стал самым молодым сотрудником компании
    \item Участие в разработке тестовой документации (тест-кейсы, чек-листы, отчеты о дефектах)
    \item Создание и поддержка автоматизированных тестов для web-приложений с использованием Python и Selenium WebDriver
    \item Проведение тестирования API с использованием Postman и разработка документации для API
    \item Создание документации для тестирования безопасности и проведение тестирования на уровне API
    \item Развертывание тестовых сред с использованием Docker и Kubernetes
  \end{itemize}

  \item \textbf{Стажер}, ООО "Новые Облачные Технологии" ("МойОфис") \hfill 1.08.2022 -- 1.09.2023
  \begin{itemize}
    \item Освоение методологий тестирования и лучших практик обеспечения качества ПО
    \item Участие в тестировании производительности офисных приложений
    \item Получил лучший результат на тестовом задании среди всех кандидатов
  \end{itemize}
\end{itemize}

% Навыки
\section{Навыки}
\begin{itemize}[leftmargin=*]
  \item \textbf{Технические навыки:} Ручное и автоматизированное тестирование,
   разработка веб-приложений, создание Telegram-ботов и WebApp,
   Парсинг данных,
   настройка и поддержка CI/CD, разработка и интеграция API,
   развертывание и конфигурация приложений,
   настройка серверов и облачных сервисов,
   настройка VPN серверов и прокси

  \item \textbf{Языки программирования:} Python, C++, GO, SQL, JavaScript/TypeScript, HTML/CSS, bash (C, Assembly, Swift)

  \item \textbf{Инструменты и технологии:} React, Node.js, Telegram Bot API, Selenium WebDriver, Postman, Docker, Kubernetes, GitLab, Jenkins, Git, Jira, Linux, Windows, MacOS

  \item \textbf{Языки:} Русский (родной), Английский
\end{itemize}

% Проекты
% \section{Проекты}
% \begin{itemize}[leftmargin=*]
%   \item \textbf{[Название проекта]} \hfill [Год]\\
%   [Краткое описание проекта]\\
%   \textit{Технологии:} [Используемые технологии]

%   \item \textbf{[Название проекта]} \hfill [Год]\\
%   [Краткое описание проекта]\\
%   \textit{Технологии:} [Используемые технологии]
% \end{itemize}

% % Сертификаты
% \section{Сертификаты}
% \begin{itemize}[leftmargin=*]
%   \item \textbf{[Название сертификата]}, [Организация] \hfill [Год]
%   \item \textbf{[Название сертификата]}, [Организация] \hfill [Год]
% \end{itemize}

% % Дополнительная информация
% \section{Дополнительная информация}
% \begin{itemize}[leftmargin=*]
%   \item \textbf{Интересы:} [Ваши интересы]
%   \item \textbf{Волонтерство:} [Ваш опыт волонтерства]
% \end{itemize}

\end{document}